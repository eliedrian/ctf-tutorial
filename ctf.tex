\documentclass[17pt]{beamer}
\usecolortheme{owl}

\usepackage{parskip}
\usepackage{tabularx}

\let\oldtexttt\texttt
\renewcommand{\texttt}[1]{{\color{pink}{\oldtexttt{#1}}}}

\title{CTF}
\subtitle{\texttt{\\flag\{shortandquick\}}}

\begin{document}

\frame{\titlepage}

\section{Categories}

\frame{\sectionpage}

\begin{frame}{Cryptography}
  ``Secret writing''

  \texttt{Base64}...???

  \texttt{XGZsYWd7aGVyZSBpdCBpc30=}
\end{frame}

\begin{frame}{Steganography}
  Hiding data in plain sight
\end{frame}

\begin{frame}{Web exploitation}
  Analyzing web requests

  ``Inspect source...''
\end{frame}

\begin{frame}{Reverse engineering}
  Analyzing binaries or programs

  \texttt{hexdump}, \emph{Ghidra}
\end{frame}

\begin{frame}{Forensics}
  Combing through data

  Examining logs, PCAPs, iso
\end{frame}

\begin{frame}{Binary exploitation}
  Buffer overflows
\end{frame}

\begin{frame}{Miscellaneous}
  General puzzles, logic problems, ``common sense'', \emph{OSINT}
\end{frame}

\section{Tools}
\frame{\sectionpage}

\subsection{Linux terminal}
\frame{\subsectionpage}

\begin{frame}{\texttt{ls}}
  List stuff(?)

  Prints out files in directory

  Each directory has a special \texttt{.} (current dir) and \texttt{..} (parent directory) file
\end{frame}

\begin{frame}{\texttt{ls}}
  ``Switches'' (\texttt{-a}, \texttt{-b}, etc.) change behavior

  \begin{tabularx}{\framewidth}{lX}
    \texttt{ls}, \texttt{ls .} & list files in current directory \\
    \texttt{ls ./exploit} & list files in directory named \texttt{exploit} \\
    \texttt{ls -a} & list \emph{all} files in current directory \\
    \texttt{ls -l} & list files in list format \\
  \end{tabularx}

\end{frame}

\begin{frame}{\texttt{mkdir}}
  Makes a directory
\end{frame}

\begin{frame}{\texttt{mkdir}}
  \begin{tabularx}{\framewidth}{lX}
    \texttt{mkdir temp} & makes a directory named \texttt{temp} \\
    \texttt{mkdir -p a/b} & makes a directory \texttt{b} in \texttt{a} (and creates \texttt{a} if it doesn't exist)
  \end{tabularx}
\end{frame}

\begin{frame}{\texttt{cd}}
  Change directory

  ``Moves'' current directory into a new one
\end{frame}

\begin{frame}{\texttt{cd}}
  \begin{tabularx}{\framewidth}{lX}
    \texttt{cd exploit} & move into a directory called \texttt{exploit} \\
    \texttt{cd ..} & move up a directory \\
    \texttt{cd a/b} & move into a directory called \texttt{b} inside of \texttt{a} \\
    \texttt{cd /} & move into the root directory
  \end{tabularx}

  \texttt{cd ..} can be chained: \texttt{cd ../../../..} moves 4 directories up
\end{frame}

\begin{frame}{\texttt{cp}}
  Copy files

  \alert{Careful!} You may lose data.
\end{frame}

\begin{frame}{\texttt{cp}}
  \begin{tabularx}{\framewidth}{lX}
    \texttt{cp a b} & copy \texttt{a} to a file \texttt{b} \\
    \texttt{cp a b c/} & copy \texttt{a}, \texttt{b} to a dir \texttt{c/} \\
    \texttt{cp -r dir1/ dir2/} & copy directory \texttt{dir1/} and its files to \texttt{dir2/}
  \end{tabularx}

  \texttt{cp} creates a copy OR overwrites
\end{frame}

\begin{frame}{\texttt{mv}}
  Move files

  \alert{Careful!} You may lose data.
\end{frame}

\begin{frame}{\texttt{mv}}
  \begin{tabularx}{\framewidth}{lX}
    \texttt{mv a b} & move \texttt{a} to a file \texttt{b} \\
    \texttt{mv a b c/} & move \texttt{a}, \texttt{b} into a dir \texttt{c/} \\
    \texttt{mv dir1/ dir2/} & move directory \texttt{dir1/} and its files into \texttt{dir2/}
  \end{tabularx}

  \texttt{mv} moves (i.e., renames) OR overwrites
\end{frame}

\begin{frame}{\texttt{rm}}
  Removes files

  \alert{Careful!} You may lose data.
\end{frame}

\begin{frame}{\texttt{rm}}
  \begin{tabularx}{\framewidth}{lX}
    \texttt{rm exploit.txt} & removes a file \texttt{exploit.txt} in the current dir \\
    \texttt{rm a b c} & remove files \texttt{a}, \texttt{b}, and \texttt{c} \\
    \texttt{rm -r haxx/} & recursively removes all files in \texttt{haxx/} and itself \\
  \end{tabularx}
\end{frame}

\end{document}

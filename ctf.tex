\documentclass[17pt]{beamer}
\usecolortheme{owl}

\usepackage{parskip}
\usepackage{tabularx}
\usepackage{hyperref}
\usepackage{booktabs}

\let\oldtexttt\texttt
\renewcommand{\texttt}[1]{{\color{pink}{\oldtexttt{#1}}}}

\title{CTF}
\subtitle{\texttt{\\flag\{shortandquick\}}}

\begin{document}

\frame{\titlepage}

\section{Categories}

\frame{\sectionpage}

\begin{frame}{Cryptography}
  ``Secret writing''

  \texttt{Base64}...???

  \texttt{XGZsYWd7aGVyZSBpdCBpc30=}
\end{frame}

\begin{frame}{Steganography}
  Hiding data in plain sight
\end{frame}

\begin{frame}{Web exploitation}
  Analyzing web requests

  ``Inspect source...''
\end{frame}

\begin{frame}{Reverse engineering}
  Analyzing binaries or programs

  \texttt{hexdump}, \emph{Ghidra}
\end{frame}

\begin{frame}{Forensics}
  Combing through data

  Examining logs, PCAPs, iso
\end{frame}

\begin{frame}{Binary exploitation}
  Buffer overflows
\end{frame}

\begin{frame}{Miscellaneous}
  General puzzles, logic problems, ``common sense'', \emph{OSINT}
\end{frame}

\section{Tools}
\frame{\sectionpage}

\subsection{Linux terminal}
\frame{\subsectionpage}

\begin{frame}{\texttt{ls}}
  List stuff(?)

  Prints out files in directory

  Each directory has a special \texttt{.} (current dir) and \texttt{..} (parent directory) file
\end{frame}

\begin{frame}{\texttt{ls}}
  ``Switches'' (\texttt{-a}, \texttt{-b}, etc.) change behavior

  \begin{tabularx}{\framewidth}{lX}
    \texttt{ls}, \texttt{ls .} & list files in current directory \\
    \texttt{ls ./exploit} & list files in directory named \texttt{exploit} \\
    \texttt{ls -a} & list \emph{all} files in current directory \\
    \texttt{ls -l} & list files in list format \\
  \end{tabularx}

\end{frame}

\begin{frame}{\texttt{mkdir}}
  Makes a directory
\end{frame}

\begin{frame}{\texttt{mkdir}}
  \begin{tabularx}{\framewidth}{lX}
    \texttt{mkdir temp} & makes a directory named \texttt{temp} \\
    \texttt{mkdir -p a/b} & makes a directory \texttt{b} in \texttt{a} (and creates \texttt{a} if it doesn't exist)
  \end{tabularx}
\end{frame}

\begin{frame}{\texttt{cd}}
  Change directory

  ``Moves'' current directory into a new one
\end{frame}

\begin{frame}{\texttt{cd}}
  \begin{tabularx}{\framewidth}{lX}
    \texttt{cd exploit} & move into a directory called \texttt{exploit} \\
    \texttt{cd ..} & move up a directory \\
    \texttt{cd a/b} & move into a directory called \texttt{b} inside of \texttt{a} \\
    \texttt{cd /} & move into the root directory
  \end{tabularx}

  \texttt{cd ..} can be chained: \texttt{cd ../../../..} moves 4 directories up
\end{frame}

\begin{frame}{\texttt{cp}}
  Copy files

  \alert{Careful!} You may lose data.
\end{frame}

\begin{frame}{\texttt{cp}}
  \begin{tabularx}{\framewidth}{lX}
    \texttt{cp a b} & copy \texttt{a} to a file \texttt{b} \\
    \texttt{cp a b c/} & copy \texttt{a}, \texttt{b} to a dir \texttt{c/} \\
    \texttt{cp -r dir1/ dir2/} & copy directory \texttt{dir1/} and its files to \texttt{dir2/}
  \end{tabularx}

  \texttt{cp} creates a copy OR overwrites
\end{frame}

\begin{frame}{\texttt{mv}}
  Move files

  \alert{Careful!} You may lose data.
\end{frame}

\begin{frame}{\texttt{mv}}
  \begin{tabularx}{\framewidth}{lX}
    \texttt{mv a b} & move \texttt{a} to a file \texttt{b} \\
    \texttt{mv a b c/} & move \texttt{a}, \texttt{b} into a dir \texttt{c/} \\
    \texttt{mv dir1/ dir2/} & move directory \texttt{dir1/} and its files into \texttt{dir2/}
  \end{tabularx}

  \texttt{mv} moves (i.e., renames) OR overwrites
\end{frame}

\begin{frame}{\texttt{rm}}
  Removes files

  \alert{Careful!} You may lose data.
\end{frame}

\begin{frame}{\texttt{rm}}
  \begin{tabularx}{\framewidth}{lX}
    \texttt{rm exploit.txt} & removes a file \texttt{exploit.txt} in the current dir \\
    \texttt{rm a b c} & remove files \texttt{a}, \texttt{b}, and \texttt{c} \\
    \texttt{rm -r haxx/} & recursively removes all files in \texttt{haxx/} and itself \\
  \end{tabularx}
\end{frame}

\begin{frame}{\texttt{cat}}
  Concatenate---prints out files' contents
\end{frame}

\begin{frame}{\texttt{cat}}
  \begin{tabularx}{\framewidth}{lX}
    \texttt{cat hello} & prints out contents of \texttt{hello} \\
    \texttt{cat a b c} & prints out \texttt{a}, followed by \texttt{b}, then \texttt{c}
  \end{tabularx}
\end{frame}

\begin{frame}{\texttt{less}}
  Similar to \texttt{cat}, but ``smarter''

  Contents can be ``paged'' and searched
\end{frame}

\begin{frame}{\texttt{less}}
  \begin{tabularx}{\framewidth}{lX}
    \texttt{less exploit} & shows contents of \texttt{exploit}
  \end{tabularx}

  Up, Down, Home, End to scroll pages

  Search with \texttt{/}\emph{\texttt{<query>}}\texttt{<Enter>}
\end{frame}

\begin{frame}{\texttt{>}}
  Take output and inserts into a file

  Output of left side > file

  \alert{Careful!} You may lose data.
\end{frame}

\begin{frame}{\texttt{>}}
  \begin{tabularx}{\framewidth}{lX}
    \texttt{cat a > b} & takes output of \texttt{cat a} and puts it into file \texttt{b} \\
    \texttt{ls -l > files} & takes file list as a list (\texttt{-l}) and saves it into a file called \texttt{files}
  \end{tabularx}
\end{frame}

\subsection{Useful UNIX tools}
\frame{\subsectionpage}

\begin{frame}{\texttt{strings}}
  Prints out ``readable'' strings
\end{frame}

\begin{frame}{\texttt{strings}}
  \begin{tabularx}{\framewidth}{lX}
    \texttt{strings a.exe} & prints out human-readable strings of \texttt{a.exe} \\
    \texttt{strings lib.png > temp} & saves strings from \texttt{lib.png} into \texttt{temp}
  \end{tabularx}
\end{frame}

\begin{frame}{\texttt{file}}
  Determines file type from contents

  Makes good guesses based on magic bytes
\end{frame}

\begin{frame}{\texttt{file}}
  \begin{tabularx}{\framewidth}{lX}
    \texttt{file a.png} & prints out determined file type (probably PNG image data) \\
    \texttt{file out} & checks contents and makes a guess at \texttt{out}'s type
  \end{tabularx}
\end{frame}

\begin{frame}{\texttt{grep}}
  Searches for patterns in files
\end{frame}

\begin{frame}{\texttt{grep}}
  \begin{tabularx}{\framewidth}{lX}
    \texttt{grep 'aBc' xyz} & searches for \texttt{aBc} in \texttt{xyz} \\
    \texttt{grep -i 'flag' a} & searches for \texttt{flag} (case-insensitive) in \texttt{a} \\
    \texttt{ls | grep -i 'dir'} & searches output of \texttt{ls} for \texttt{dir} (case-insensitive)
  \end{tabularx}
\end{frame}

\begin{frame}{\texttt{xxd}}
  Prints out file bytes in hex format

  ASCII column on the right
\end{frame}

\begin{frame}{\texttt{xxd}}
  \begin{tabularx}{\framewidth}{lX}
    \texttt{xxd a.png} & prints out \texttt{a.png}'s bytes \\
    \texttt{xxd -c 32 a.png} & same as above, but double the columns
  \end{tabularx}
\end{frame}

\begin{frame}{\texttt{objdump}}
  Disassemble executables

  Prints out machine code instructions
\end{frame}

\begin{frame}{\texttt{objdump}}
  \begin{tabularx}{\framewidth}{lX}
    \texttt{objdump -d a.exe} & disassembles and prints out assembly code of \texttt{a.exe} \\
    \texttt{objdump -d a.exe > disas} & same as above, but save to file \texttt{disas}
  \end{tabularx}
\end{frame}

\subsection{Web tools}
\frame{\subsectionpage}

\begin{frame}{CyberChef}
  \url{https://gchq.github.io/CyberChef/}

  Swiss-army knife to handle data

  Learn to recognize patterns in data
\end{frame}

\begin{frame}{CyberChef}
  \small
  \begin{tabularx}{\framewidth}{lX}
    \texttt{ZmxhZ3t0ZXN0fQ==} & \emph{looks} like \texttt{Base64}, use 'From Base64' recipe \\
    \midrule
    \texttt{flag\%7Burl\_encoded\%7D} & URL Decode \\
    \midrule
    \texttt{66 6c 61 67 7b 68 65 78 7d} & From Hex \\
    \midrule
    \texttt{Idk} & Try 'Magic' recipe \\
  \end{tabularx}
\end{frame}

\begin{frame}{Browser dev tools}
  ``Inspect element...''

  \texttt{<Ctrl><Shift><I>}
\end{frame}

\end{document}
